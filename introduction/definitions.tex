\section{Definitions}
\subsection{Dropout}
Dropout is defined as a student who starts university studies but for some reason does not finish his degree. 
Dropout can be caused by any number of reasons such as health problems, personal reasons, family and other work taking up too much time.
It is therefore hard to define when a student has given up. 
He or she might for example go back to school when they have resolved the family issues.
In our specific case we define student dropout as a student who does not finish his computer science studies.
If a student switches studies to for example electrical engineering he is still considered as a dropout student.
This is mainly because the data we have been able to look at until now only contains data about computer science students.
We do still need a way to classify the student as a dropout or not for our training data.
One approach taken by ~\cite{7} is to classify students able to finish their first year courses in 3 years as successful,
everyone else is considered unsuccessful.
In the future someone might have to manually mark each student in the training data as successful or not,
but this cannot be decided at this time.

\subsection{Student Data}
Student data is defined as all the data we or the paper being discussed have available on our students.
This varies a lot in different cases.
Some studies in dropout prediction only use university grades and number of attempts on these courses.
Others have extensive numbers on different tests,
like study efficiency scores and other social experiments they have run on the students. 
In our case we have student grades for all courses,
number of tries on these courses,
personal information like name, sex and age.
However we do not have any data except the obvious data stored in any database.
For example we do not have grades from high school and students take no kind of study tests when they start at NTNU.
There is neither any data from it's learning which is the learning management system NTNU uses. 

