\section{Background and motivation}
	The main motivation for this project is to provide the institutes and faculties at NTNU 
	with the necessary tools to identify students who requires extra motivation, and what the biggest obstacle for these students are.
	This will be done in an effort to reduce the relative high drop out rate which has been
	seen in norwegian education.
	Research published in 2013 by OECD showed that only 59\% of the students
	starting at one of the higher edcucation institutes in Norway completed their study.
	In comparison to a 68\% average when looking at the other 34 OECD countries\cite{OECD2013}.
	Studies done by NTNU itself shows an even worse statistic for four IT-studies at IME~\footnote{Faculty of Information Technology, Mathematics and Electrical Engineering}
	which shows an average of 57\%\cite{ntnu:dropout}.

	\bigskip\noindent
	Futher motivation comes from the fact that the educational institutes in Norway collects
	copius amounts of data for each student. Though necessary in order to keep track of the progression, 
	this data is to a large extend idle. And is not used in order to further develop the quality of education. 
	This project will serve as an introductary analysis of the possible analyses of this data, 
	and take a deep dive into the possible emergent information this data can hold. 
	
\subsection{How dropout is handled today}
At NTNU, specifically the computer science department there are no measures taken to pinpoint students in danger of dropping out.
Even if they were found, students in danger of departing from college is not handled in any particular way or offered special support today.
Neither do they utilize the potential of all the data stored in student data system(FS).
There is a severe lack of integration into the faculty society before the 7th semester.