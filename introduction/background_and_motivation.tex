\section{Background and motivation}
	The main motivation for this project is to provide the institutes and faculties at NTNU 
	with the necessary tools to identify struggling students, and what their struggles are.
	This will be done in an effort to mitigate the relative high drop out rate which has been
	seen in norwegian education. 
	In 2013 only 59\% of the students starting at one of the higher educational institutes completed their study.
	In comparison to a 68\% average when looking at the other 34 OECD countries.\cite{OECD2013}
	
	Futher motivation comes from the fact that the educational institutes in Norway collects
	copius amounts of data for each student. Though necessary in order to keep track of the progression, 
	this data is to a large extend idle. And is not used in order to further develop the quality of education. 
	This project will serve as an introductary analysis of the possible analyses of this data, 
	and take a deep dive into the possible emergent information this data can hold. 
	
\subsection{How dropout is handled today}

	