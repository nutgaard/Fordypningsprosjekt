\section{Background and motivation}
	The main motivation for this project is to provide the institutes and faculties at NTNU 
	with the necessary tools to identify students who requires extra motivation, and what the biggest obstacle for these students are.
	This is of interest due to how educational institutes in Norway are financed, 
	where throughput of students directly influence the budget for the years to come\cite{finansiering}. 
	It would therefore be profitable to reduce the relative high drop out rates seen in Norwegian Education, 
	thus increasing funding, education more student, and enabling the educational institutes to improve their quality even further.
	
	\bigskip\noindent
	Research published in 2013 by OECD showed that only 59\% of the students
	starting at one of the higher education institutes in Norway completed their study.
	In comparison to a 68\% average when looking at the other 34 OECD countries\cite{OECD2013}.
	Studies done by NTNU itself shows an even worse statistic for four IT-studies at IME~\footnote{Faculty of Information Technology, Mathematics and Electrical Engineering},	which shows that just an average of 57\%\cite{ntnu:dropout} complete their study within the normal timeframe.
	Given these numbers it seems clear that the Norwegian educational system have room for some major improvements. 

	\bigskip\noindent
	Further motivation comes from the fact that the educational institutes in Norway collects
	copious amounts of data for each student in the student data system, \textbf{FS\footnote{Felles systemet}}. 
	Though necessary in order to keep track of the progression, 
	this data is to a large extend idle. And is not used in order to further develop the quality of education. 
	This project will serve as an introductory analysis of the possible analyses of this data, 
	and take a deep dive into the possible emergent information this data can provide. 
	
\subsection{How dropout is handled today}
At NTNU, specifically the Department of Computer and Information Science(IDI) there are no measures taken to pinpoint students in danger of dropping out.
Even if they were found, students in danger of departing from the university is not handled in any particular way or offered special support today.
The attrition rate is much larger than desirable and the blame is put on students switching to other faculties after one year.
The faculty wants to raise the quality of applicants without lowering the student limit, and thus an discussion on student retention is also provided. 
%Neither do they utilize the potential of all the data stored in student data system(FS).
%There is a severe lack of integration into the faculty society before the 7th semester.