\subsection{Student retention theory}
The importance of limiting student attrition is a field widely researched.
The dominant model in research until recently is Tintos model of student retention~\cite{1982}.

\bigskip\noindent
Tinto's model is based on Durkheim's theory of suicide~\cite{Tinto01031975}.
Tinto creates a link between the problems of student retention and suicide by looking at an educational institution as a local society with its own norms, values and social structures.
The same methods Durkheim uses in his discussion of suicide in a normal society, 
Tinto then applies to dropout in the local society that exists withing a university, 
or any institution of higher education.
The main link between suicide and dropout is a lack of integration of an individual into the fabric of society.
A lack of integration into the system will lead to low commitment and eventually departure from college.
The proposed model therefore looks at departure from college as a function of the interplay between an institution and an individual.
It is heavily based on the interaction between faculty and student and how this interaction should be increased and improved.

\bigskip\noindent
Later Tinto published a paper on the limits of his model, other theories and proposed practices in student attrition~\cite{1982}.
In this work he discusses multiple reasons for dropping out but not all of them are applicable to our situation. For example financial problems, which we can't consider a significant reason in Norway as we have a great support system for anyone who wants to study as well as free education.
He also looks at solutions which have to be implemented at a national level, but as we limit ourselves to one university in our discussion this is not relevant.
However on the institutional level there are many things that are directly relatable.

\bigskip\noindent
As mentioned, lack of integration to the institutions society is the main problem in student dropout.
This manifests itself in first year students, it is often a result of unrealistic expectations about social life as well as the academic level.
Much can be gained in this area by marketing realistically, and not just to get as many applicants as possible.
The social aspect however is not handled by marketing, and the shock of moving to a new city from a relatively safe environment should also be addressed.
At NTNU there is a great social integration system, but it is mainly handled by student organizations which offer sponsors to show groups of students around the school and city.
Even though this offer exists, the faculty can do a lot to improve the social integration on their side including student-lecturer relations.

\bigskip\noindent
There are of course limits to what the institution can do.
Freshmen start at these institutions with varying skills, abilities, interests and commitment to complete a given program of study.
It is useful to group students by these factors as every retention measure is quite specific.
The most interesting group of students to focus retention measures on are the committed ones with sufficient preexisting knowledge to complete the degree.
These students are also more likely to change studies instead of dropping out all together.
There is also the issue of reducing dropout vs reducing the quality of the study.
As there are certain goals the university works towards, reducing quality is normally not an option and it is shown that program quality rather than general appeal is the key to most faculties effectiveness.

\bigskip\noindent
The institution can improve on the following things.
They can provide consistent and frequent advisement, especially in the first year,
provide better social integration,
host informal interactions with other students and the faculty outside the classroom,
stimulate the students interests and provide them with the rewards they seek in higher education.
In general the more interest the faculty shows in its students the higher the retention rate will be.
It should be noted that institutions which have implemented such measures not only improved their attrition but also attracted a greater amount of students the following years.

\bigskip\noindent
Often the problem of implementing such measures boils down to lack of means.
NTNU receives funding based on how many students complete their degree and the discussed methods are therefore highly relevant as they can pay of very fast.
It is therefore important to properly distribute these means on measures that will be most effective and for the students who need it the most.
Though it might look like it, it's not all about earning money.
NTNU's vision is: Knowledge for a better world.
They wish to educate students to tackle important challenges faced by Norway and the world community.
As such they want as many students as possible to succeed and go out after a finished degree and be an innovative force for society. 

\bigskip\noindent
There will always be people who discover that higher education is not for them.
Not everyone is motivated enough and find other ways to spend their time and attention, this is not easy to predict or worth trying to prevent.
In the end none of the measures discussed can eliminate dropout completely but it has been shown numerous times that any institute can do a lot, within reason, to improve attrition. 
The question remains of course how much this will cost in regards to the rewards of such actions.

\bigskip\noindent
ATTRACT
This project focuses on attracting and retaining students withing science and technology studies without lowering the standards. 

