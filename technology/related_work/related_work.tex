\section{Related work}
Explaining and predicting academic performance has been researched a lot in other fields than computer science. Many studies look at the psychological aspect of students that choose to quit studying or are preforming under the standards. Not many of these existing techniques take all the data available on students into account. Data mining in this context can therefore be very helpful. Not only by organizing the data into an easier analyzable way for counselors, but also by trying to classify and analyze both students and education in general. Recently a lot of research has gone into this and the term Educational Data Mining~\cite{1} was born.

\bigskip\noindent
Educational Data Mining includes everything about data mining in an educational context.
We are especially interested in what classification algorithms have to offer. 
Classification in EDM can be used in many ways, such as:
providing feedback for supporting instructors,
make recommendations for students,
predict student performance,
detect undesirable student behaviours,
constructing course content and custom tutoring,
analyze social networks and much more.
It is still not clear which methods are most effective in this context, but most papers so far agree that decision trees~\cite{2,12} or rule implying algorithms are the best approach.
It is not yet proven, albeit we need an understandable way to represent the data found, and rule based or decision tree algorithms give us exactly this. 

\bigskip\noindent
An interesting paper writes about identifying cellular phone failures~\cite{3}.
Their task is not to predict any failure but to identify possible causes that resulted in failures.  
They have used a technique called Class Association Rules which only mines rules with classes on the right side of the implication. 
Many different data mining techniques have been discussed and why none of them worked in a real data situation. 
A lot of the points they bring up here applies to our task as well.
Class Association Rule mining is further explained in~\cite{4}.

\bigskip\noindent
A lot of papers have the same main problem as us, predicting dropout prone students.~\cite{5,7,8,9,11} 
This is an important area of research as many universities use a lot of time, 
money and effort in educating students who will not complete their degree. 
These education spots could even be occupied by more focused students. 
Some of these students drop out because of external events, but some might be able to finish with the right help. 
It is therefore important to find these students to either help them or drop them from the study. 

\bigskip\noindent
One of the papers is very similar to our task in almost every way~\cite{7}.
They look at students in an electrical engineering program. 
They classified a successful student as having done all his 1 year courses in 3 years. 
This is more or less accurate but there are outliers of course. 
They run some tests on 3 datasets, one with university data only, one with pre-university data only and one combination of both. 
Only one is relevant to us, the dataset containing university grades only, since we will not have access to earlier data. 
They then ran different decision tree algorithms on the data with 10 fold cross validation. 
The only university data used here was grades and number of tries in 74 different courses. 
With only the highest grades for the 37 available courses and how many attempts were taken pr course.
The algorithms had an accuracy between 75 and 80\% on dropout which is very promising for us.

\bigskip\noindent
Some other, often more statistical papers talk about the importance of social data. 
This can be in the form of data from student conversations with counselors.~\cite{11} 

\bigskip\noindent
A data mining tool has been developed for Moodle(Moodle is a Learning Management System similar to its-learning).~\cite{12} 
They use this mining tool to compare different data mining techniques when classifying students based on their Moodle usage. 
This might be interesting but not many lecturers or students use its-learning efficiently. 