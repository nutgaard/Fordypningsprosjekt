% F�rst spesifiserer vi hvilken dokumentklasse vi vil ha og noen 
% globale opsjoner. Bytt ut 'article' med 'book' hvis du vil ha 
% med kapitler.
\documentclass[a4paper, twoside, titlepage, 11pt]{article}

% S� sier vi fra om hvilke tilleggspakker vi trenger
% til dokumentet v�rt. De som du ikke trenger (se kommentaren) 
% kan det v�re en fordel � kommentere ut (sett prosenttegn foran),
% da vil kompilering g� raskere.

\usepackage[norsk]{babel}             % norske navn rundt omkring
\usepackage[T1]{fontenc}              % norsk tegnsett (���)
\usepackage[latin1]{inputenc}         % norsk tegnsett
\usepackage{geometry}                 % anbefalt pakke for � styre marger.

\usepackage{amsmath,amsfonts,amssymb} % matematikksymboler
\usepackage{amsthm}                   % for � lage teoremer og lignende.
\usepackage{graphicx}                 % inkludering av grafikk
\usepackage{subfig}                   % hvis du vil kunne ha flere
                                      % figurer inni en figur
\usepackage{listings}                 % Fin for inkludering av kildekode

%\usepackage{hyperref}                % Lager hyperlinker i evt. pdf-dokument
                                      % men har noen bugs, s� den er kommentert
                                      % bort her.
                                 
% Indeksgenerering er kommentert ut her. Ta bort prosenttegnene
% hvis du vil ha en indeks:
%\usepackage{makeidx}     
%\makeindex              

% Selve dokumentet begynner:

\begin{document}

% P� forsida skal vi ikke ha noen sidenummerering:

\pagestyle{empty}
\pagenumbering{roman}

% Inkluder forsida:
% Enkel forside som bruker latex sin \titlepage kommando:
% NB: Bruken av \and mellom navn!
\titlepage
\title{Lebesgueintegrering}
\author{Leonhard Euler Jr. \and Karl Fr. Gauss Jr.}
\date{2. mai 2002}
\maketitle

% Local Variables:
% TeX-master: "master"
% End:


% Romerske tall p� alt f�r selve rapporten starter er pent.
\pagenumbering{roman}

% For � ikke begynne innholdslista p� baksida av forsida:
\cleardoublepage
% (kun aktuelt n�r man har twoside som global opsjon)

% N� vi vil ha noe i topp- og bunnteksten
\pagestyle{headings}

% Si til LaTeX at vi vil ha ei innholdsliste generert akkurat her:
\tableofcontents

% Pass p� at neste side ikke begynner p� baksida av en annen side.
\cleardoublepage

% Arabisk (vanlige tall) sidenummerering. Starter p� side 1 igjen.
\pagenumbering{arabic}

% Inkluder alle de andre kildefilene:

% NB: Vi trenger ikke ta med filendelsen .tex her. Den vet
%     LaTeX om selv!

\input{innledning}

% Dette eksempelet er laget for article-dokumentklassen. Hvis 
% skriver i 'book'-dokumentklassen vil du kanskje bytte ut 
% \section med \chapter, \subsection med \section, osv...

\section{Bakgrunnsteori}

bla.bla bla

\subsection{Integralets opprinnelse}

bla bla bla

\subsection{Riemann-integralet}

bla bla bla


% Local Variables:
% TeX-master: "master"
% End:


\input{resultater}

% Bibliografi/referanseliste skal komme f�r appendiks
\bibliography{kurs}
\bibliographystyle{plain}

% En latex-kommando for � si fra at kapitlene/seksjonene fra n� 
% av skal nummereres med store bokstaver:
\appendix

\input{appendiks}

% Indeks for rapporten. Ta bort prosenttegn hvis du vil ha det med.
%\printindex

% Avslutter dokumentet v�rt:
\end{document}

% Local Variables:
% TeX-master: "master"
% End:
