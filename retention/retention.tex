\chapter{Student attrition and retention}
Student attrition and retention are two very similar and widely researched topics. 
Attrition means the reduction of the size of a workforce by not replacing personnel lost. 
In our case this means the gradual reduction of students at a university because of students departing from college, while new students normally only enter in the first or fourth year. 
Retention is the act of retaining. 
In other words to keep possession of or hold in place. 
Here we discuss university student retention, 
specifically how faculties can hold on to students which would otherwise transfer to another faculty, college or leave all together.

\bigskip\noindent
We see that attrition and retention are basically two sides of the same coin. 
Studies on attrition talk about why students choose to leave school while studies on retention talks about how to keep students at a university. 
Both are often mentioned in the same works. 

\section{Tinto's model}
The dominant model for student attrition and retention research is Tinto's model of student retention(Figure:~\ref{fig:tinto}). 
This is a theoretical model that explains the process of interaction between the individual and the institution that lead differing individuals to drop out from institutions of higher education~\cite{Tinto01031975}. 
A severe problem in similar research is that no one differentiate between dropout resulting from academic failure and voluntary withdrawal. 
Few distinguishes between permanent dropout and temporary dropout or transfer to other institutions. 
The model therefore distinguishes between the processes that result in different forms of dropout behavior.  

\image{images/tinto.png}{\columnwidth}{Tinto's model of student retention}{fig:tinto}

\bigskip\noindent
This theoretical model has roots in Durkheim's theory of suicide.
Tinto creates a link between the problems of student retention and suicide by looking at an educational institution as a local society with its own norms, values and social structures.
The same methods Durkheim uses in his discussion of suicide in a normal society, 
Tinto then applies to dropout in the local society that exists within any institution of higher education.
The main link between suicide and dropout is a lack of integration of an individual into the fabric of society.
A lack of integration will lead to low commitment and eventually departure from college.
The proposed model therefore looks at departure from college as a function of the interplay between an institution and an individual.
The purpose of the model is to be a platform on which we can discuss different interaction studies and how they be improved.

\bigskip\noindent
Later Tinto published a paper on the limits of his model, other theories and proposed practices in student attrition~\cite{1982}.
In this work he discusses multiple reasons for dropping out but not all of them are applicable to our situation. For example financial problems, which we can't consider a significant reason in Norway as we have a great support system for anyone who wants to study as well as free education.
He also looks at solutions which have to be implemented at a national level, but as we limit ourselves to one university in our discussion this is not relevant.
However on the institutional level there are many things that are directly relatable.

\bigskip\noindent
As mentioned, lack of integration to the institutions society is the main problem in student dropout.
This manifests itself in first year students.
It is often a result of unrealistic expectations about social life as well as the academic level.
Much can be gained in this area by marketing realistically, and not just to get as many applicants as possible.
The social aspect however is not handled by marketing, and the shock of moving to a new city from a relatively safe environment should also be addressed.
At NTNU there is a great social integration system, but it is mainly handled by student organizations which offer sponsors to show groups of students around the school and city.
Even though this offer exists, the faculty can do a lot to improve the social integration on their side including student-lecturer relations.

\bigskip\noindent
There are limits to what the institution can do.
Freshmen start at these institutions with varying skills, abilities, interests and commitment to complete a given program of study.
It is useful to group students by these factors as every retention measure is quite specific.
The most interesting group of students to focus retention measures on are the committed ones with sufficient preexisting knowledge to complete the degree.
These students are also more likely to change studies instead of dropping out all together.
There is also the issue of reducing dropout vs reducing the quality of the study.
As there are certain goals the university works towards, reducing quality is normally not an option and it is shown that program quality rather than general appeal is the key to most faculties effectiveness.

\bigskip\noindent
An institution can improve on the following things.
They can provide consistent and frequent advisement, especially in the first year.
Provide better social integration,
host informal interactions with other students and the faculty outside the classroom,
stimulate the students interests and provide them with the rewards they seek in higher education.
In general the more interest the faculty shows in its students the higher the retention rate will be.
It should be noted that institutions which have implemented such measures not only improved their attrition but also attracted a greater amount of students the following years.

\section{NTNU}
A study on education quality has been done at the faculty of information technology, mathematics and electrical engineering at NTNU~\cite{ntnu:dropout}. 
Here the education quality of the faculty is discussed. 
A problems discovered with student retention is that many students enter this faculty because it has lower requirements than other studies, and really wish to switch to another faculty as soon as possible. 
This is possible because once you are a student at NTNU nearly all the engineering degrees require the same courses the first year. 
After a year you can switch faculties without having to do extra work. 
If there are unique courses you can simply sign up for these as well, thereby ``abusing`` the system.

\bigskip\noindent
It is desirable to raise the quality of IKT-programs applicants. 
An ambition proposed is that the IKT-programs should raise its applicant quality above the average of other engineering programs at NTNU.
Problems discovered from the statistics that needs attention are:
Compared to how many spots there are and the number of applicants, there are too many students who drop out. 
The IKT-programs cannot compete with other technology programs when it comes to attracting students. 
Of the total amount of applicants too few actually want to study at IME.
Many students get very bad grades the first semester compared to other programs.
Tinto's studies address exactly these problems among others. His proposed methods would aid greatly with all these problems and in the proposed ambition.

\bigskip\noindent
Often the problem of implementing such measures boils down to lack of means.
NTNU receives funding based on how many students complete their degree and the discussed methods are therefore highly relevant as they can pay of very fast.
It is therefore important to properly distribute these means on measures that will be most effective and for the students who need it the most.
Though it might look like it, it's not all about earning money.
NTNU's vision is: Knowledge for a better world.
They wish to educate students to tackle important challenges faced by Norway and the world community.
As such they want as many students as possible to succeed and go out after a finished degree and be an innovative force for society. 
With this vision it is also a shame that the students who drop out are taking an important spot that someone more motivated could fill.

\bigskip\noindent
In order to properly distribute the means the institution has available we must find the students who need it the most.
NTNU should identify these students to limit the waste of time, effort and money on students which does not necessarily need extra attention.
Even if no support systems for these students are built it can be useful for them to get a heads up.
They might not realize that they are in a danger zone of dropping out.

\bigskip\noindent
As mentioned earlier NTNU have no systems to detect students in the danger of dropping out.
Neither do they utilize any of the data they have on students, except the obvious tasks for it like signing students up for courses etc. 
In this paper we discuss multiple ways we can utilize this data in an educational decision support system.
It has been shown that many data mining techniques have an accuracy of around 80\% on classifying dropout students.
Other systems visualize the data in efficient ways for estimating student performance or course structures. 
In general a lot can be done with this idle data and time, effort and money can be saved, if not earned.

\bigskip\noindent
There will always be people who discover that higher education is not for them.
Not everyone is motivated enough and find other ways to spend their time and attention, this is not easy to predict or worth trying to prevent.
It is a shame that the programs offered at IME attracts a majority of these students, but efforts should be put into making the programs more attractive instead of helping these students. 
In the end none of the measures discussed can eliminate dropout completely but it has been shown numerous times that any institute can do a lot, within reason, to improve attrition. 
The question remains of course how much this will cost in regards to the rewards of such actions.