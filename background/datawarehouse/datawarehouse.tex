\section{Datawarehouse}
	\noindent
	This section will provide an introductory overview over the datawarehousing discipline and commonly terminology within. 
	For the non-database gurus the datawarehousing dicispline is, for all intent and purposes, an organization scheme of
	data created to enable and facilitate data analyses without the absolute focus on normalization.
	%Though not the only key features that differentiate datawarehouses from ordinary databases.
	\cite{wiki:normalization, wiki:datawarehouse}
	
	\bigskip\noindent
	In the datawarehouse field you will often find buzz words describing commonly known principles from ordinary database
	usage. Some of the most used include \textit{facts table}, \textit{measures} and \textit{dimensions}.
	These words describe to a large extends the concepts around designing a datawarehouse. 
	
	\bigskip\noindent
	\textit{Measures} are exactly what the avid reader may intuitively allready know, quantitative data about a business.
	While \textit{dimensions} can be considered to be meta-data, and works as descriptive attributes. 
	As an example from the service industry, one can consider a storechain and each stores sales. 
	In this case one could look at each sale as a fact with \textit{price} as possible \textit{measure},
	and \textit{store location} as a \textit{dimension}. 
	The difference between \textit{measures} and \textit{dimensions} become more apparent when considering their usage.
	One common queryform would be to use an arithmetic function, as \texttt{average} or \texttt{sum}, over a set of
	\textit{measures} filtered by a value from the \textit{dimensional} set.
	The \textit{facts table} is the combination of these two, and hence holding all the "`facts"' related to 
	
	
	\bigskip\noindent
	In the further discussion of datawarehouses we will look at two prominent organizational schemes utilizes 
	within the datawarehouse discipline, namely \textit{star schema} and \textit{snowflake schema}. 
	Both of which have gotten their names by their resemblance to their respective natural phenomena.
	Common for the both of them is that the \textit{facts table} serves as the main storage table in the scheme, 
	and is often pictured as the center of the warehouse in graphical illustrations (see figures below). 
	
	\bigskip\noindent
	Since the \textit{star schema} is the simplest, and considered a special case of the \textit{snowflake scheme}
	we will start off with explaining the basics around this schema. 
	As mentioned in the paragraph above it often
	
	
	
	
	The typical datawarehouse utilizes a extract-transform-load process (ETL), 

	