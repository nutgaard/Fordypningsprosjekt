\subsection{Association analysis}
	The technique of association analysis consists of discovering interesting relationships between variables
	in a database~\cite{Piateski:1991:assocrules,tan:2006:datamining,Agrawal:1993:assocrules}.
	Again looking at a store chain as an example, the administration wants to increase optimize the placement of products throughout their stores. 
	In this case association analysis can help by looking at what items are bought together. 
	Then by looking at the association rules it would be possible to increase sales and profit by placing associated items close to each other.
	Previous work within educational datamining has looked at how information stored at e-learning-platforms\footnote{It's learning, Moodle, etc.}
	can be analyzed to determine student performance\cite{merceron2008interestingness}. 
	This work can be adapted to use other data sources, and thus find other interesting relationships,
	as relationships between different courses.  
		
	\bigskip\noindent
	In order to proficiently describe the task of association analysis we borrow the notation from Tan et.al~\cite{tan:2006:datamining}.
	The problem definition, as per Tan et.al, is defined by; 
	the binary attributes $I = i_1, i_2, \ldots, i_d$, 
	transactions $T = t_1, t_2,\ldots, t_N$, where each transaction	is represented as a binary vector $t$, with $t[k]$ representing the state of $I_k$ in $t$, and the support count $\sigma (X) = |{t_i|X\subseteq t_i, t_i \in T}|$
	An association rule is thus defined as $X \Rightarrow Y$, where $X$ and $Y$ are disjoint itemsets.
	
	\bigskip\noindent
	From the definition above there are several useful properties of each rule, most known is perhaps \textit{confidence} and \textit{support},
	where $support$ is a measurement of the proportion of transactions in the dataset which contain the itemset. 
	Thus a low support value for any given rule indicates that the given rule may simply be a result of coincidence. 
	While $confidence$ yields an estimate for $P(Y|X)$ given the all the transactions.
	\begin{figure}[H]
		\begin{align}
			support(X\Rightarrow Y) &= \frac{\sigma (X \cup Y)}{N}\nonumber\\
			confidence(X\Rightarrow Y) &= \frac{\sigma (X \cup Y)}{\sigma (X)}\nonumber\\
		\end{align}
	\end{figure}
	
	\bigskip\noindent
	One should however be careful when analizing these results as to not fall into the pits of the \textit{post hoc, ergo propter hoc\footnote{After this, therefore, because of this}}-fallacy\cite{posthoc}. Whereby mistaking correlation for causality.
	
	