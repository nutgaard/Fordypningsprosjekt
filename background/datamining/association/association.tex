\subsection{Association analysis}
	The technique of association analysis consists of discovering interesting relationships between variables
	in a database~\cite{Piateski:1991:assocrules,tan:2006:datamining,Agrawal:1993:assocrules}.
	In order to proficiently describe the task of association analysis we borrow the notation from Tan et.al~\cite{tan:2006:datamining}.

	\bigskip\noindent
	The problem definition, as per Tan et.al, is defined by; 
	the binary attributes $I = i_1, i_2, \ldots, i_d$, 
	transactions $T = t_1, t_2,\ldots, t_N$, where each transaction	is represented as a binary vector $t$, with $t[k]$ representing the state of $I_k$ in $t$, and the support count $\sigma (X) = |{t_i|X\subseteq t_i, t_i \in T}|$
	An association rule is thusly defined as $X \Rightarrow Y$, where $X$ and $Y$ are disjoint itemsets.
	
	\bigskip\noindent
	From the definition above there are several useful properties of each rule, most known is perhaps \textit{confidence} and \textit{support},
	where $support$ is a measurement of the proportion of transactions in the dataset which contain the itemset. Thus a low support value for any given rule indicates that the given rule may simply be a result of coincidence. $confidence$ yields an estimate for $P(Y|X)$ given the all the transactions.
	\begin{figure}[H]
		\begin{align}
			support(X\Rightarrow Y) &= \frac{\sigma (X \cup Y)}{N}\nonumber\\
			confidence(X\Rightarrow Y) &= \frac{\sigma (X \cup Y)}{\sigma (X)}\nonumber\\
		\end{align}
	\end{figure}
	
	\bigskip\noindent
	One should take care when analysis these results as not not fall into the pits of the \textit{post hoc, ergo propter hoc\footnote{After this, therefore, because of this}}-fallacy\cite{posthoc}. Whereby mistaking correlation for causality.
	
	