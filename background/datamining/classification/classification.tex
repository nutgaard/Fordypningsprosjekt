\subsection{Classification}
The classification problem consists of building a model capable of determining
the class/category of new observations on the basis of previously seen data. 
Training of a classifier is done by feeding it data where the class of the all the instances are known.
Once completed the classifier has created a model describing the training data, 
and can therefore determine the class of new and unique instances.
For this specific project the usage of classification will be to determine whether or not a 
given student is prone to dropping out and why. 
We therefore want an algorithm that correctly classifies the outcome for a given student, 
but also an algorithm that can shine some light into why students do not complete their
study. 

%\begin{figure}
%	\begin{align}
%		instance &= \langle f_1, f_2, \ldots , f_n, Class\rangle\nonumber\\
%		I &= \langle i_1, i_2, \ldots , i_n\rangle\nonumber\\
%		
%	\end{align}
%\end{figure}

%The methods described as classification methods are mostly tweaked algorithms taken from the field of artificial intelligence. 
%We will start by describing some of the more promising algorithms for our problem and data.
\subsubsection{Classification and regression tree}
	Classification and regression tree (\textbf{CART}) is an umbrella term used to refer to
	classification trees and regression trees.\cite{trees:umbrella}

	\bigskip\noindent The \textbf{CART} classifiers are a simple, yet powerful, classifier based around the tree structure. 
	They are often used within the datamining discipline in order to create a predictive model based for some preexisting data.
	The predictive model can then be used on new never seen before data in order to predict the most probable classification or value. 
	
	\bigskip\noindent Classification trees and regression trees are for all intent and purposes very similar, but some key differences are worth noteing. 
	One of these differences is in the outcome of their predictive model. 
	Classification trees, as the name may suggest, tries to pinpoint which \textit{class} an instance may belong to. 
	Whereas with regression trees the outcome of the model is a \textit{real number}, as the price of a certain item in a store. 
	
	\bigskip\noindent
	An example of decision tree applications can be seen as trying to predict the rating(1-10) of wine. 
	Given a dataset(Resource: \cite{mining:datasetexample}) of preexisting knowledge, which include known attributes to a set of wines and their acoompanying rating.
	A decision tree may be constructred to generalize this knownledge in order to give a prediction of other wines. 	
		
	\image{images/TrivialDecisionTree.png}{null}{Example classification tree from TDT4171}{fig:decisiontree}
	
	\bigskip\noindent
	One example of a widely used CART algorithm comes from Ross Quinlan, the inventor of the ID3, C4.5 and C5.0 algorithms. \cite{quinlan:id3, quinlan:c45}
	All of whom use the notion of entropy in order to select the attribute to split for at a certain point in the tree. 
	They all use entropy(equation~\ref{eq:entropy}) as a measurement of information gain(equation~\ref{eq:gain}) by selecting a certain attribute at a given time during the algorithm. 
	The higher the gain is, the better do the attribute serve as an indicator of an instances' classification. (Algorithm~\ref{alg:id3})
	\begin{figure}[H]
		\begin{align}
			H(S) &= -\sum_{x \in X}p(x)log_2p(x)\label{eq:entropy}\\
			IG(A) &= H(S) - \sum_{t \in T}p(t)H(t)\label{eq:gain}\\
			\intertext{where}
			S &: \text{The current data set}\nonumber\\
			X &: \text{The set of classes in $S$}\nonumber\\
			p(x) &: \text{The proportion of the numbfer of elements in class $x$ to $S$}\nonumber\\
			H(S) &: \text{The entropy of $S$}\nonumber\\
			T &:	 \text{The subset created from splitting $S$ by $A$}\nonumber
		\end{align}
		\caption{The mathematics behind entropy and information gain.}
	\end{figure}
	
%\subsubsection{Regression Classifiers}
	
\subsubsection{Artificial Neural Network}
	Another technique from the discipline of artifical intelligence is artificial neural networks (\textbf{ANN}).
	Though ANNs may be used for a variaty of task, we will focus on what ANNs are and how they may be used as
	a classifier. 
	
	\bigskip\noindent
	First of, a general introduction to ANN is in order. ANNs are computation models that attempt to capture
	the behaviour and adaptive features of the brain by modeling it.
	An ANN therefore consists of a set of neurons, the computational unit, and their synapses, the relationship between neurons (see figure~\ref{fig:ann}).
	
	\image{images/ann.png}{0.5\columnwidth}{Structure of artifical neural network}{fig:ann}
	
	\bigskip\noindent
	Neurons is characterized by a set of parameters, including the connection strength, threshold, and an activation function
	(see figure~\ref{fig:neuron} and equations~\ref{eq:annsummation}-\ref{eq:annsigmoid}). 
	This is however just the most common and basic implementation of a neural network. 
	The full discussion and other extended versions of neural networks are beyond the scope of this paper.
	
	\begin{figure}
		\begin{align}
			net_j &= \sum_{i=1}^n x_i*w_{ij}\label{eq:annsummation}\\
			o(j) &= \varphi (net_j - \theta_j)\label{eq:annthreshold}\\
			\varphi(x) &= kx\label{eq:annlinear}\\
			\varphi(x) &= \begin{cases}
					1, & \text{if $x > \theta_j$}.\\
					0, & \text{otherwise.}
				\end{cases}\label{eq:annbinary}\\
			\varphi(x) &= \begin{cases}
					1, & \text{if $x > \theta_j$}.\\
					-1, & \text{otherwise.}
				\end{cases}\label{eq:annbipolar}\\
			\varphi(x) &= \frac{1}{1+e^{-kx}}\label{eq:annsigmoid}
		\end{align}
	\end{figure}
	
	\image{images/neuron.png}{\columnwidth}{The inner workings of a artifical neuron}{fig:neuron}
	
	\bigskip\noindent
	Training of neural networks consists of three known paradigms, supervised, unsupervised and reinforced. 
	Where it is supervised learning that is of interest to us in the context of a classification task. 
	Unsupervised learning could however be used as a clustering technique, 
	while reinforce learning often is modeled as a Markov decision process and therefore fits the task of sequential decision making. \cite{bioAI}
	
	\bigskip\noindent
	Within the supervised learning paradigm there exists several algorithms. 
	Most commonly known are perhaps \textit{Widrow-Hoff} rule (delta rule)\cite{widrowhoff}, and the later extended version \textit{backpropagation of error}\cite{rumelhart1986learning}\cite{rumelhart1986parallel}. 
	The core of the backpropagation algorithm consists of calculating how each of the nodes contribute to the final answers' error, 
	and adjusting the weights of those nodes according to a preset learningrate. 
	What this algorithms in a sense does it to preform a gradient descent search of the error function at each neuron.
	Thus making the neural network as a whole converge towards the target function. 
	
	\bigskip\noindent
	One disadvantage of the ANN approach is that the resulting network in most cases yields little to none addtional information.
	This stands in contrast to the CART based classifers, where the final classifier (tree) in most cases give addtional interpretable information. 
	One example of this would be to look at the root node of a tree generated by one of the entropy based algorithms.
	The root node would in those cases represent the attribute which best classified the trainingset, 
	and thus the attribute which has be greatest influence of the target classification. 
	This additional analysis would be extremely tedious to perform using artifical neural networks due to their highly connected nature.
	Research into rule extraction from trained neural network has been done with satisfying result, 
	it still requires an extra step of postprocessing to elict the rules.~\cite{augastarule}
	
	
	
	
	
	
\subsubsection{Bayesian Network Classifiers}
	The bayesian classifier are based probabilistic classifiers which uses Bayes theorem as a basis(Equation~\ref{eq:bayes}). 
	Bayesian networks, often referred to as belief networks in AI, 
	is a representation of conditional dependencies using a directed acyclic graph(Figure~\ref{fig:bayesiannetwork}).
	This representation is choosen over full joint probability distribution table(FJPDT) due to its simplicity in designing and that it requires much less probabilistic values. 
	
	\begin{figure}[H]
		\begin{align}
			P(A|B) &= \frac{P(B|A)P(A)}{P(B)}\label{eq:bayes}
		\end{align}
	\end{figure}
	
	\image{images/bayesiannetwork.jpg}{\columnwidth}{Example of bayesian network}{fig:bayesiannetwork}
	
	\bigskip\noindent
	As an example, given an network of $n$ boolean variables, each with $k$ parent nodes. Using a FJPDT it would require $2^n$ probabilities, 
	whereas a bayesian network would make due with $n*2^k$. 
	Since $k$, in most circumstances, are very small compared to $n$ this would result in just a fraction of the probabilites.
	And since every probability from the FJPDT can be calculated from a bayesian network, no data is lost. 
	
	\bigskip\noindent
	The final nail in coffin for the FJPDT comes from the difficulty of estimating the probabilites. 
	E.g The probability $P(Y|\vec{X})$ where $\vec{X}$ may contain several attributes that do not affect the probability. 
	This is often cumbersome and difficult is the table is to be populated by a human domain expert.
	
	\bigskip\noindent
	A special case of bayesian network is the \textit{naive Bayes}, which is based upon a naive independence assumption between attributes. Thus making the somewhat complicated calculations seen in bayesian network become much simpler.
	
	\begin{figure}[H]
		\begin{align}
			P(Y, X_1, \ldots ,X_n) &= P(Y)P(X_1, \ldots ,X_n|Y)\nonumber\\
					&= P(Y)P(X_1|Y)P(X_2, \ldots ,X_n|C,X_1)\nonumber\\
					&= P(Y)P(X_1|Y)P(X_2, \ldots ,X_n|C,X_1)\ldots P(X_n|C,X_1,\ldots,X_{n-1})\nonumber
			\intertext{the naive conditional independence assumptions}
		  P(X_i|C,X_j) &= P(X_i|C)\nonumber\\
			\intertext{for all $i \neq j$, thus giving}
			P(Y, X_1, \ldots ,X_n) &= \frac{1}{Z}P(Y)\prod_{i=1}^n P(X_i|C)
		\end{align}
	\end{figure}
	
	\image{images/naivebayes.png}{\columnwidth}{Naive bayes structure}{fig:naivebayes}
	
	
	
	
