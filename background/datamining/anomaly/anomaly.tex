\subsection{Anomaly detection}
	Anomaly detection is the process of identifying (and possibly removing) objects which do not conform to the expected pattern or model. 
	Many approaches to anomaly detection include the use of 
	previously mentioned techniques from clustering, as model-based, proximity-based and density-based techniques. 
	
	\bigskip\noindent
	\textbf{Model-based} techniques tries to generalize data into a model, 
	f.ex a probability distribution, clusters, regression model etc. 
	An object is then considered an anomaly if it does not fit the model in some way. 
	
	\bigskip\noindent
	\textbf{Proximity-based}, or distance-based, techniques are based around proximity or distance in a n-dimensional euclidean space. 
	Where object that has a low proximity or high distance to other object may be considered anomalies.
	
	\bigskip\noindent
	\textbf{Density-based} techniques uses an estimate of the density surrounding each object in order to classify wether or not the given object can be viewed as normal compared to the complete dataset. Objects in low density regions are located where few other objects are located, hence they may be viewed as anomalies given the whole dataset.
	
