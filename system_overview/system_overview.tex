\chapter{Project proposition}
	For the general structure of a education datamining system we propose 
	a six step process~(fig:~\ref{fig:steps}). 
	The first three should be recognized from the chapter on data warehousing. 
	Step four is an intermidiate step which falls outside the realm of datawarehousing and datamining. 
	The motivation behind this step is to enable flexibility in the datastructure, since classifiers may need different structured data.
	Step five is the datamining step where Weka will be used as a basis
	in order to analyse the data. 
	The final step will revolve around the presentation of results from the analysis and other general visual data mining techniques. 
	
\image{images/steps.png}{\columnwidth}{Dataflow}{fig:steps}

%\section{Problems and decisions}
\subsection{Data preprocessing}
Anonymizing data in a collection of data is very important when dealing with sensitive data. 
Especially if it is for data that is going to be published. 
A lot of times even if you remove name, id and so on there might be a certain combination of attributes that yield an induviduals identity.
An approach called \textit{k-anonymization} which maintains the classification structure is proposed, 
which basically makes it so that at least $k$ instances contains every combination of linked attributes. 
A top-down approach to refine the data maximizing the trade-off between information and anonymity is presented.~\cite{14}

\bigskip\noindent
We might need to discretize our data.
We can put grades into buckets, for example \textless C, C and \textgreater C into buckets called FAIL, PASS and GOOD.~\cite{12}

\bigskip\noindent
Reducing the number of attributes, in our case the number of courses is also important for understanding and space in the database.
It has been shown that not only can classification properties of the data be preserved, but the accuracy can increase for certain algorithms as well~\cite{9}.
This is also important because students can take different courses so some of these attributes may not be very relevant. 
In the worst case it might even cause false results or conclusions in the classification process.

\bigskip\noindent
One of the biggest issues in many papers is that the dataset is not balanced. 
Data is imbalanced when one class is represented by a significantly larger amount of instances than another~\cite{10}. 
An imbalanced boolean dataset might look like this: class 1 is represented by 10 instances and class 0 is represented by 200 instances. 
Learning a classifier from such datasets poses a problem. 
A class that is poorly represented will yield bad accuracy for this class in the classifier, 
even though the accuracy for the dataset as a whole might be high. 
This is an important issue in our discussion as our dataset is imbalanced and the poorly represented class is the most important one, 
we want to find students which will fail.

\bigskip\noindent
As mentioned the accuracy might be high overall but not for the spesiffic classes we want.
A better measure for classifiers built on imbalanced datasets is the geometric mean. 
This is usually used in these circumstances~\cite{12}. 

\bigskip\noindent
Balancing out the dataset can be done in multiple ways. 
Random over-sampling copies random instances of an underrepresented class until the classes are represented by an equal amount of instances~\cite{12}. 
Synthetic Minority Over-sampling Technique(SMOTE) introduces new minority instances synthetically from the nearest neighbour of equal class~\cite{9}. 
Cost-sensitive classification lets you put weights on the classes so the classifier is built to maximize the accuracy based on these weights~\cite{9}. 
We can specify that fail students are 5 times as important to classify correctly then pass students. 
When changing the dataset for classification it is important to test on the original dataset if you are testing on and building from the same dataset.

\bigskip\noindent
The task of mining on imbalanced datasets is further explained in~\cite{10}.

\bigskip\noindent
They also discuss existing data mining tools for imbalanced datasets in~\cite{8}.

\bigskip\noindent
All these papers have different datasets and focuses on results but none of them take all the concerns stated in the other into account. 
Some might discretize attribute values but not take into account the imbalanced datasets and visa versa.

\subsection{Educational Decision Support System}
An educational decision support system has been proposed~\cite{5} to optimize effort, time and money for both educators and students. 
This focuses on finding the students that will not be able to finish their studies and who should be given a second chance. 
The DDS focuses on making reasonable decisions in student management. 
They want to pinpoint poor study performance, they manage a warning list of who will be forced to stop their study. 
Which students that don't have the necessary knowledge to make it in the next semester. 
Students may get one extra change semester to get back in the game. 
If the wrong students are given a new chance time, money and effort will be lost. 
With over 1348 undergrad students they got a 97\% student classification accuracy.

\bigskip\noindent
A second paper also discusses the problem of making correct academic decisions.~\cite{6} 
Different ways of analyzing performance data is discussed and PADSS is developed, a software package. 
Need to evaluate students in some way. The importance of the data extracted from the grades databases has been shown. 

\subsection{Visualization of data}
\section{Extract}
We must extract the data from the FS system(initially this will be the only source of data).
In FS every exam a student has signed up for is listed.
For each exam there is a grade and a code which explain the circumstances of the exam.
For example a student may have 3 attempted exams in one course, two fail grades with the classifications delivered blank and did not finish enough assignments and one exam with the grade C.
Since the database structure and accessibility is not confirmed, research on this step is limited.
However we have seen examples of student data being extracted through CSV files. 
This is still a necessary step and it should be seen as the process of loading the data from a given resource into memory, and thus preparing it for the transformation process.

\bigskip\noindent
In order to achieve a higher accuracy we should also investigate 
the possibility of extracting data from It's learning, and a module within FS called \textbf{SO\footnote{Samordna opptak}} which handle student application, to augment the FS data.

\section{Transform}
In order to get the data persisted in the data warehouse we need to transform the data.
We do still need a way to classify whether the student got a degree or dropped out for our training data.
One approach taken by ~\cite{7} is to classify students able to finish their first year courses in 3 years as successful,
everyone else is considered unsuccessful.
If a method for classification such as this or other assumptions and rules can be agreed upon the process can be done automatically.
Accuracy might be increased though if a domain expert manually classifies the training data.

\bigskip\noindent
A second step of the transformation phase is anonymization. 
This is normally not part of the standard process, but due to the possibly sensitive nature of our data this is a necessary step.
We need to anonymize the data by removing obvious fields such as name, address, student id and so forth. 
Student can however still be identified by their grades, hence removing unique entries from the data set may be an option.
K-anonymization, anomaly detection and common clustering techniques may help us achieve this goal while maintaining the classification structure and accuracy.

\section{Load}
The transformed data from the previous phase is persisted into the datawarehouse. 
In this specific case this data may include courses, grades, exams, number of tries for each course, 
and any reason a student may have for not completing a course or failing an exam.

\bigskip\noindent
We propose an exam centric data model, where exam entries are treated as the \textit{fact table}. 
The reason behind choosing an exam centric model instead of a student centric model comes from the
need to keep the dimensions of each table at a fixed size. Something that would not be possible 
with the student centric model due variations in courses and number of tries at each course.
\image{UML/examCentricDataModel.png}{\columnwidth}{The general structure of an exam centric data model}{fig:examCentricDataModel}

\noindent
The proposed data model still yields the same flexibility as a student centric model would. 
But may see some performance penalties due to the extra \textit{table joins} that are needed to consolidate the data.
This performance decrease can however be mitigated by the use of \textit{materialized view}.
For example a view over number of tries each student has on each of his passed courses along with the best obtained grade in the respective course.

\bigskip\noindent
We suggest the use of SQL for the database. The choice of dialect and vendor is not relevant but MySQL is free under the GNU General Public License and will integrate nicely with a java system.

\section{Prepare}
Although a lot of data juggling is done during the transformation phase, it does not specialize the data towards its final use. 
The prepare phase aims to close the final gap between the persisted data structure and the optimal data structure for the classifiers.
Depending on the classifier to be used several approaches may be of relevance, especially balancing of training data to be used.
The balancing is there because of the skewed probability distribution of dropout given a random student attending the school.
Thus students who complete their study will dominate the classifier and yield poor misclassification accuracy for cases where students quit.

\bigskip\noindent
Another point of attention is discretization of values. 
This is due to the specific needs of different classifiers, where some does not handle continuous values well. 
One attribute that may be interesting to discretize is the age of a given student, 
since age group may be a better indicator of a students performance than the age itself. 
Other aggregated values, as average grade, may also gain accuracy by discretization.

\bigskip\noindent
Another factor in choosing Weka as the framework is its built-in features for both of the techniques mentioned above. 

\section{Analyze}
As seen in chapter~\ref{ch:related} the most efficient data mining techniques for educational data are the simple and explanatory ones,
specifically decision trees and rule inducing algorithms.
Artificial neural networks have also been used with varying success and should be considered.
Serge Herzog's paper~\cite{2} examined the prediction accuracy of several data-mining methods in an educational context.
It found that the level of complexity of the data used is the main factor in determining which method is better.
On average decision trees and neural networks perform at least as good as the well-established logistic regression models in the case of student retention.
In larger data sets, data-mining algorithms worked notably better as well as providing explanatory results. 
Neural networks worked better than decision trees and rule induction methods in complex studies, like estimating degree completion time for new and transfer students simultaneously. 
It is also mentioned that adding multiple hidden layers and pruning underused neurones can greatly increase accuracy. 
However there is the tradeoff between accuracy and explanatory results to consider. 

\bigskip\noindent
Gerben Dekker~\cite{7} compares decision trees, a bayesian classifier, a logistic model, a random forest method and a rule-based learner.
The conclusion is that simple classifiers provide more useful results while providing accuracies between 75 and 80\%.
Even though they have a balanced data set they did not end up with a satisfactory accuracy for  false negatives(dropout students classified as successfull). As mentioned this is the important measure for us as we wish to identify students at risk of leaving the university. Cost sensitive learning is used instead of balancing the data set to prefer one kind of misclassification to another and is strongly advised in addition to or instead of data balancing.  

\bigskip\noindent
If the focus is more on finding courses that make students quit or similar student behavior across courses, association analysis should be strongly considered. This technique may yield significant results regardless and should be experimented with. It provides statistical evidence of useful rules with its measures, support and confidence. In chapter~\ref{ch:relclass} a method called Class Association Rules, which only mines rules with class as the implied attribute can be used if only the courses that make students quit are of interest or performance becomes an issue.

\bigskip\noindent
In any case Weka is the dominant software to use. 
It is a free java library and it provides all the techniques we have discussed in this paper, and can be extended without any trouble.
It also comes with a very intuitive classification, preprocessing and data flow editor through its GUI.

\section{Visualize}	
Visualization is the step of communicating the information found clearly through graphical means. 
This is useful because humans are exceptional at extracting structures from pictures compared to computers. 
Integration between visualization and data mining is called visual data mining.
In this proposed system we have chosen a loose integration of visual data mining where output from the analyze step is used as input for the visualization step.
Other possible integrations are no integration, where you only use one of the methods or full integration, where both methods are applied in parallel.
Visualization techniques that might be interesting include histogram, pie chart, scatter plots, line graphs, icon based methods and pixel based methods.
The choice is largely dependent on the number of dimensions and whether variables are related to each other or not.

\bigskip\noindent
This is helpful for understanding which attributes are most significant for student dropout. 
Not only is it helpful, but many mining techniques require user intervention at different stages and visualization is starting to be used to support the decision process~\cite{1207445}.
Overall visual data mining looks promising and this should be expanded on in the future.
