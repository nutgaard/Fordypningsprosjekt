\section{Overview of system}
\subsection{Extract}
We must extract the data from the FS system. We do not still know how this is done, but in some cases they can create an excel document with the needed information.
You should also think about extracting data from It's learning.

\subsection{Transform}
In order to get the data from the excel documents into the data warehouse we need to preprocess it first. This can be done manually or automatic if you know the format.
We need to anonymize the data by removing obvious fields such as name, address and so on.
The sex of a person might be of high relevance, but we can probably not use this either.
K-anonymization will also be helpfull.
\subsection{Load}
	\image{UML/examCentricDataModel.png}{\columnwidth}{The general structure of an exam centric data model}{fig:examCentricDataModel}
Student data is defined here as any data we can get from the FS system. 
This system contains all the information on every student as NTNU.
There are grades, number of tries at an exam, reason for failing at an exam, 
We need to put this data into a data warehouse. We have made a proposed model for how this could look.
\subsection{Prepare}
Depending on what kind of analysis you are gonna run, or if you are only going to visualize it you might need to preprocess the data from the datawarehouse again.
We suggest using java and calling the functions from the Weka library, and from here you can also get the data from the data warehouse easily.
\subsection{Classify/Analyze}
Using Weka run the algorithms and methods described.
\subsection{Visualize}
Visualization of data.